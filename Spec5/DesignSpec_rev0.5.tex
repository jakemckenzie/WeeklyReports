\documentclass[english,12pt]{article}

\usepackage[english]{babel}
\usepackage[utf8]{inputenc}
\usepackage[T1]{fontenc}
\usepackage{hyperref}
\usepackage{siunitx}
\usepackage{graphicx}
\usepackage{tabularx}
\usepackage{longtable}
\usepackage[hmargin=1in,vmargin=1in]{geometry}
\usepackage[dvipsnames]{xcolor}
\usepackage[normalem]{ulem}
\usepackage{flafter} 
\useunder{\uline}{\ul}{}
\begin{document}

\begin{center}

\thispagestyle{empty}

$ $

\vspace{250pt}

\begin{bfseries}

{\Large Object Recognition and Path Smoothing Robot, Phase 5}

{\Huge Design Specification}

%{\Large $\langle$application and version to be tested$\rangle$}%

\end{bfseries}

\vspace{180pt}

University of Washington Tacoma, School of Engineering and Technology

%TIE-21204 Ohjelmistojen testaus%

\vspace{12pt}

Authors: 

Ammon Dodson \href{mailto:ammon0@uw.edu}{ammon0@uw.edu} 

Alex Marlow \href{mailto:alexmarlow117@gmail.com}{alexmarlow117@gmail.com} 

Jake McKenzie \href{mailto:jake314@uw.edu}{jake314@uw.edu}

Distribution: Matthew Tolentino

Document state: draft

Modified: \today

\end{center}

\newpage


\tableofcontents

\newpage


\section{Revision History}

\begin{itemize}
	\item v0.1.0: Initial specifications for the document itself.
    \item v0.2.0: Additional information was added about the Robot 
    Operating System (ROS).
    \item v0.2.1: Ethical concerns were added to the document.
    \item v0.2.2: Created a diagram for the overall system architecture.
    \item v0.3.0: Additions to the introduction. Rewrite of the requirements section with
    enumeration. Additions to the System Architecture and System Design
    section including hardware architecture and software architecture with
    higher level and lower level diagrams with text describing these diagrams
    and text for these sections generally.
    \item v0.4.0: Made the requirements table. Clarified some requirements. Improved page
    formatting. Added some to the design section. Use case diagram created for
    the introduction and general mistakes addressed.
\end{itemize}


\section{Introduction}
\subsection{Overview}
This design specification describes the architectures and design framework that will aid and abet the
production of the ORPS-Robot version 0.4 – The Object Recognition and Path Smoothing Robot. The
ORPS-Robot will be a platform for validating the research of Michael McCourt and a scheme for
exploring object recognition via OpenCV with Robot Operating System, which is a powerful framework
for writing robot software. There will be a demonstration of Simultaneous Localization and Mapping
(SLAM). Together this will demonstrate a “finder robot” with applications in search and rescue and
threat detection.\\\\
Additionally, there will be beacon triangulation and/or GPS to fuse additional location information into
the SLAM or finder functionality. Typically, feedback is used when some amount of uncertainty exists
within a system and the environment for which that system operates. To implement feedback this
requires analyzing the underlying dynamics of whatever system you plan to look at and the
implementation of communications, computing and software to accomplish some task. The project aims
at investigating the dynamics of a semi-autonomous robot with the Object Recognition and Path
Smoothing Robot by delivering a platform that is some interplay of communications, computing and
software to Michael McCourt to aid in his research in control.\\\\
The networked control system to be explored in this project is commonly used in telerobotics.
Telerobotics is still in the experimental phase of research and has been theorized for active shooter
detection, a way for doctors to diagnose patients in disaster areas and telecommute from home.
\subsection{Network Control}
Network control systems have many useful applications; The typical use case involves using robots to
interact with an environment that is too hazardous for a person. Any such Network control involves
some delay of both outgoing control signals and incoming sensor data. In some situations, this delay
may impair the intended function of the network control system.\\\\
Dr. McCourt has developed a set of filters intended to be placed in such a delayed, closed-loop control
system. These filters apply a mathematical transformation on both incoming and outgoing loop signals
such that communication delays are mitigated. This specification outlines the development of a
controller-robot system intended to demonstrate the McCourt filter.
\subsection{Autonomous Control}
There may also be use cases for robots in hazardous environments where direct human control is
impossible. Such a robot must be able to autonomously navigate and interact with an unknown space.
SLAM is a fundamental technology for such autonomous activity, allowing the robot to navigate.
Additionally, such an autonomous machine must be able to sense and recognize an objective before
being able to interact with it. Computer vision is another
fundamental technology for sensing a real-world environment.
Sensing a condition is a necessary first step for being able to act
based on the current environment.
\subsection{Scope}
This specification covers the following:\\
\begin{itemize}
    \item Any and all hardware modifications to the ORPS-Robot.
    \item Software installed on the burger bot insofar as it deviates
    from the stock installation.
    \item Base station control software setup insofar as it deviates
    from stock installation.
    \item Any necessary techniques for integrating a Human
    Interface to the base station installation.
    \item Any methods used to implement a communication delay
    between the robot and its base station.
    \item Implementation of the McCourt input output transformation.
    \item An overview of implementing SLAM on the ORPS-Robot.
    \item Integration of OpenCV into the ORPS-Robot.
\end{itemize}
\section{Requirements}
In this section we will delve through the minimum requirements of this project to translate the needs of
our client Michael McCourt into precise targets, establish metrics for a successful product and support
design trade-off decisions. For this revision of the design specification we will articulate the marginally
acceptable target values, without any of the ideal target values for our stretch goals.
**INSERT TABLE HERE**
\subsection{Network Control}
\begin{itemize}
    \item[R1.] \textbf{There will be some ROS capable mobile platform.} \\
    This robot must be ROS capable, mobile, and remotely controllable. The mobile platform must be
    capable of processing the R-transform for incoming and outgoing signals. All the hardware and software
    modules of the mobile platform must fit within the constraints of that platform, including such things as
    power, weight, and processing speed.
    \item[R2.] \textbf{There must be a base station interfaced with a USB game controller.} \\
    The Human controller we have available currently is a USB game controller. This controller will have to
    be connected to some base station that can connect to the ORPS-Robot wirelessly. The base station is a
    laptop which will run the graphical user interface for the user, route communication traffic, perform
    transformations and public commands for the ORPS-Robot.
    \item[R3.] \textbf{The base station must communicate with the robot through a delayed link.} \\
    In order to demonstrate the McCourt input-output transformation the robot must be remotely
    controllable. Control signals must be routed through a delayed communication medium. Ideally, or as a
    second stage, the human controller’s feedback information should also be routed through the delayed
    medium.
    \item[R4.] \textbf{The Robot Should be wirelessly controllable.} \\
    To fully demonstrate the usefulness of the McCourt transform we should model a real-world situation
    where the robot is out of sight and controllable solely from the base station.
    \item[R5.] \textbf{The robot must be able to report its position in space.} \\
    To fully demonstrate the McCourt input-output transformation the full control loop should be routed
    through it as shown in Figure 2. This will require the position feedback to be in the form of a simple
    numerical array such as an x-y coordinate.
    \item[R6.] \textbf{There must be a means of recording and comparing the planned and actual path the robot follows.} \\
    To demonstrate the McCourt input-output transformation we need to be able to compare the actual
    and planned path and have some measure of how closely they align. We should be able to make
    multiple test runs with and without the filter functioning and compare the course fidelity in aggregate.
\end{itemize}
\subsection{Autonomous Control}
\section{System Architecture}
\subsection{Hardware Architecture}
\subsection{Software Architecture}
\section{System Design}
\subsection{Hardware Design}
\subsubsection{Objectives}
\subsubsection{Constraints}
\subsubsection{Composition}
\paragraph{Xbox 360 Controller}
\paragraph{Marvelmind Indoor Navigation System}
\paragraph{Raspberry Pi Model B}
\paragraph{OpenCR}
\paragraph{DYNAMIXEL Actuator System}
\paragraph{Li-Po Battery}
\paragraph{LDS-01 LiDAR}
\subsubsection{Interface}
\paragraph{Xbox 360 Controller}
\paragraph{Marvelmind Indoor Navigation System}
\paragraph{Raspberry Pi 3 Model B}
\paragraph{OpenCR}
\paragraph{DYNAMIXAL Actuator System}
\paragraph{Li-Po Battery}
\subsection{Software Design}
\subsubsection{Objectives}
\subsubsection{Constraints}
\subsubsection{Compositions}
\subsubsection{Uses and Interactions}
\subsubsection{Interface}
\subsubsection{Resources}
\subsubsection{Base Station Software}
\paragraph{Robot Operating System (ROS)}
\paragraph{Graphical User Interface (GUI)}
\subsubsection{R-Transformation}
\subsubsection{M-Transformation}
\subsubsection{Human Interface Design}
\section{Ethical Considerations}
\section{References}
\section{Errata}
\end{document}
\documentclass[english,12pt]{article}

\usepackage[english]{babel}
\usepackage[utf8]{inputenc}
\usepackage[T1]{fontenc}
\usepackage{hyperref}
\usepackage{siunitx}

\usepackage[hmargin=1in,vmargin=1in]{geometry}


\begin{document}

\begin{center}

\thispagestyle{empty}

$ $

\vspace{250pt}

\begin{bfseries}

{\Large Object Recognition and Path Smoothing Robot, Phase 1}

{\Huge Test Plan}

%{\Large $\langle$application and version to be tested$\rangle$}%

\end{bfseries}

\vspace{180pt}

University of Washington Tacoma, School of Engineering and Technology

%TIE-21204 Ohjelmistojen testaus%

\vspace{12pt}

Authors: 

Ammon Dodson \href{mailto:ammon0@uw.edu}{ammon0@uw.edu} 

Alex Marlow \href{mailto:alexmarlow117@gmail.com}{alexmarlow117@gmail.com} 

Jake McKenzie \href{mailto:jake314@uw.edu}{jake314@uw.edu}

Distribution: Matthew Tolentino

Document state: draft

Modified: \today

\end{center}

\newpage


\tableofcontents

\newpage


\section{Revision History}

\begin{itemize}
	\item v0.1: In this version of the testplan we defined the  
    scope of the problem with the introduction, test items and 
    approach. 
\end{itemize}


\section{Introduction}
\subsection{Purpose}
This test plan describes the testing approach and overall 
framework that will drive the testing of the ORPS-Robot 
version 0.1 – The Object Recognition and Path Smoothing 
robot. The document introduces:
\begin{itemize}
	\item[] Test Strategy: These are the rules the test will e based on including 
    the givens of the project (e.g.: start / end dates, objectives, assumptions); 
    description of the process to set up a valid test (e.g.: entry / exit criteria, 
    creation of test cases, specific tasks to perform and scheduling)
	\item[] Execution Strategy: describes how the test will be performed 
    and process to identify and report problems, and to fix and implement 
    fixes.
    \item[] Test Management: process to handle the logistics of the test 
    and all the events that come up during execution (e.g.: communications, 
    escalation procedures and risk mitigation)
\end{itemize}
\subsection{Project Overview}
The ORPS-Robot will be a platform for validating the research of Michael McCourt 
and a scheme for exploring object recognition via OpenCV with Robot Operating System, 
which is a powerful framework for writing robot software. There will be a demonstration 
of Simultaneous Localization and Mapping (SLAM). Together this will demostrate a ``finder
robot'' with applications in search and rescue and threat detection. Additionally there 
will be beacon triangulation and/or GPS to fuse additional location information into the 
SLAM or finder functionality.
\subsection{Audience}
Collaborative robotics software development for research in control systems 
for path smoothing. Collaboration in academic research is usually thought to mean 
equal partnership between two academic faculty members who are pursuing mutually 
interesting and beneficial research. In our case we are creating a platform which 
will serve Dr. McCourt research where deep understanding of the control system in 
place is not require on our part and a deep understanding of the robot are not 
require on the part of Dr. McCourt.\\\\
Creating truly, robust, general-purpose robot software is hard. As 
undergraduates using robot operating system framework allows us to encompass 
solving robotics problems in real-world variations in complex tasks and 
environments that no single individual, laboratory, or institution could 
hole to create completely on their own from scratch. The audience for this 
device is us, as it serves our education.\\\\
Applications in path smoothing and object recognition used in 
ORPS-Robot project are heavily used in semi-autonomous and autonomous 
vehicles. The knowledge gain in applying these skills in the ROS ecosystem 
should serve us to gaining a greater knowledge of the systems and 
practices of both control systems and object recognition.
\section{Test Items}
\subsection{Hardware Test Items}
These will have more descriptions in later revisions.
\begin{itemize}
    \item[] \ang{360} LiDAR for SLAM \& Navigation
    \item[] Rasberry Pi 3 Model B
    \item[] 32-bit ARM Cortex-M7 OpenCR
    \item[] DYNAMIXEL wheels
    \item[] Li-Po Battery 11.1V 1,800mAh 
    \item[] Xbox 360 Controller 
    \item[] **INCLUDE SENSORS PROVIDED BY DR. MCCOURT HERE** 
\end{itemize}
\subsection{Software Test Items}
I don't know what this will look like yet and am leaving this blank for now.
\subsection{System Test Items}
I don't know what this will look like yet and am leaving this blank for now.

% \begin{itemize}
% 	\item test environment(s) and required setup
% 	\item how the testing should be performed, including usage scenarios, example users etc.
% \end{itemize}


\section{Approach}

% \begin{itemize}
% 	\item how is the severity of found errors determined
% 	\item criteria for deciding whether the SUT passes testing
% \end{itemize}

\section{Fail Criteria}

\section{Testing Deliverables}

\section{Roles}

\section{Schedule}

\section{Testing Risks and Mitigation}

\section{Approvals}

\end{document}

\documentclass[12pt]{IEEEtran}
\usepackage{hyperref}
\begin{document}
\title{Week 1 Status}
\author{Bruce Baker, Ammon Dodson, Jake McKenzie, Tim Olchondra and Matthew Skipworth}
\maketitle

So for the first week and prior the work that has been done on the senior project 
was mostly by Jake. During the fall break Jake met with Dr. Michael McCourt to help 
the professor along with a research project of his. Currently the project is being 
designed to have stages with a baseline being used to serve Dr. McCourt’s research. 
This project will have a user feed input to a  
\href{http://wiki.ros.org/Robots/TurtleBot}{Turtlebot (Burger)}.\\

\indent This user will attempt to follow a path laid out for them.  Dr. McCourt 
has a an equation to smooth out noisy inputs, and we have ideas for how we might 
extend the project to be a better mechanism to test his equation.\\

\indent The current idea for the project is to split it into three stages with the 
possibility of stages being added in the future. The first stage of the project will 
be to get the Turtlebot to be controlled by an Xbox controller. The second stage will 
be to get the sensor for the Turtlebot to talk to the other sensors placed in a room. 
The last stage will be to get a webcam attached to Turtlebot to use OpenCV so that the 
user operates the camera to follow the path.\\

\indent The current idea for the project is to split it into three stages with the 
possibility of stages being added in the future. The first stage of the project will 
be to get the Turtlebot to be controlled by an Xbox controller. The second stage will 
be to get the sensor for the Turtlebot to talk to the other sensors placed in a room. 
The last stage will be to get a webcam attached to Turtlebot to use OpenCV so that the 
user operates the camera to follow the path. \\

\noindent \href{https://www.facebook.com/jake.mckenzie.16/videos/2450973201587222/}{
Jake has completed about half of a MOOC online in ROS where he has gotten a version 
of the Turtlebot to spin.}\\

\indent During the first week the group met with Dr. McCourt to talk over the 
specifications for the project.  In this meeting we discussed a general schedule 
of product milestones. Currently the group is at five students but the groups 
will split into two groups. The three main milestones are as follows.

\begin{enumerate}
    \item Getting the Turtlebot to be controlled with both a wireless keyboard and more importantly an Xbox 360 controller.
    \item Getting the sensors that Dr. McCourt bought send and receive positioning information 
    from each other. We currently have four sensors for sending and one for receiving(which 
    will go on top of the Turtlebot)
    \item Getting the webcam to work on top of the Turtlebot with OpenCV and SLAM.
    \item Getting a neural network to reduce error in position by taking in video and 
    positioning data.
    \item Getting a neural network to be able to recognize the types of objects that the Turtlebot to avoid obstacles like people, dogs, and doors.
\end{enumerate}
\noindent \indent The first three milestones are the ``meat'' of the project while the other two milestones are the stretch goals. More milestones are being workshopped but this is the current product cycle. After step 3 the two groups will diverge exploring different milestones to be finalized later this quarter. 
\end{document}
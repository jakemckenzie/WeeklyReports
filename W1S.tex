\documentclass[12pt]{IEEEtran}
\usepackage{hyperref}
\begin{document}
\title{Week 1 Status}
\author{Bruce Baker, Ammon Dodson, Jake McKenzie, Tim Olchondra and Matthew Skipworth}
\maketitle

So for the first week and prior the work that has been done on the senior 
project was mostly by Jake. During the fall break Jake met with doctor 
Michael McCourt to talk over help the professor along with a research 
project of his. Currently the project is being designed to have stages with
a baseline being used to serve Mike's research. This project will have a user
feed input to a \href{http://wiki.ros.org/Robots/TurtleBot}{turtlebot (Burger)}.\\

\indent This user will attempt to follow a path laid out for them. Mike has an equation
to smooth out noisy inputs which he would like to test and we have ideas
for how we might extend the project to be a better mechanism to test his equation.
Jake does not quite understand how that equation works or how it will interact with
the project but that's where we are currently at.\\

\indent The current idea for the project is to split it into three stages with 
the possibility of stages being added in the future. The first stage of the
project will be to get the Turtlebot to be controlled by an Xbox controller.
The second stage will be to get the sensor for the turtlebot to talk to the sensors
we will place in a room. The last stage will be to get a webcam that is attached to
turtlebot to use OpenCV so that the user uses the camera to follow the path.\\

\indent To test the robot we will need access to a large room that we are working on finding 
and getting access to on campus. That does not need to happen immediately but it will needed 
later on. A meeting was scheduled with Mike on Monday. \\

\noindent \href{https://www.facebook.com/jake.mckenzie.16/videos/2450973201587222/}{Jake has completed 
about half of a MOOC online in ROS where he has gotten a version of
the turtlebot to spin.}\\

\indent During the first week the group met with Dr. McCourt to talk over the specifications 
for the project. Currently the group is at five students but the groups will split into
two groups of hopefully three or one group of three and one group of two. In this meeting 
we discussed a general schedule of product milestones. The three main milestones are 
as follows.

\begin{enumerate}
    \item Getting the Turtlebot Burger to be controlled with both a wireless keyboard and more importantly an Xbox 360 controller.
    \item Getting the sensors that Dr. McCourt bought to talk to talk and receive positioning information from each other. We 
    currently have four sensors for recieiving and one for talking(which will go on top of the Turtlebot Burger)
    \item Getting the webcam work on top of the Turtlebot Burger with OpenCV and SLAM.
    \item Getting a neural network to reduce error in position by taking in video and positioning data with a neural network (sensor 
    fusion, etc)
    \item Getting a neural network to be able to recogize the types of objects that the Turtlebot to avoid obstacles like people, dogs, doors, etc.
\end{enumerate}
 \indent Above is the current plan for the project after the first meeting. The first 
three milestones are the ``meat'' of the project while the other two milestones are the 
stretch goals. More milestones are being workshopped but for the time being this is the 
current product cycle. After step 3 the two groups will diverge exploring different milestones
both included and to be finalized later this quarter. 
\end{document}